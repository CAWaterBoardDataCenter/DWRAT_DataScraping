% Options for packages loaded elsewhere
\PassOptionsToPackage{unicode}{hyperref}
\PassOptionsToPackage{hyphens}{url}
%
\documentclass[
]{article}
\usepackage{amsmath,amssymb}
\usepackage{lmodern}
\usepackage{iftex}
\ifPDFTeX
  \usepackage[T1]{fontenc}
  \usepackage[utf8]{inputenc}
  \usepackage{textcomp} % provide euro and other symbols
\else % if luatex or xetex
  \usepackage{unicode-math}
  \defaultfontfeatures{Scale=MatchLowercase}
  \defaultfontfeatures[\rmfamily]{Ligatures=TeX,Scale=1}
\fi
% Use upquote if available, for straight quotes in verbatim environments
\IfFileExists{upquote.sty}{\usepackage{upquote}}{}
\IfFileExists{microtype.sty}{% use microtype if available
  \usepackage[]{microtype}
  \UseMicrotypeSet[protrusion]{basicmath} % disable protrusion for tt fonts
}{}
\makeatletter
\@ifundefined{KOMAClassName}{% if non-KOMA class
  \IfFileExists{parskip.sty}{%
    \usepackage{parskip}
  }{% else
    \setlength{\parindent}{0pt}
    \setlength{\parskip}{6pt plus 2pt minus 1pt}}
}{% if KOMA class
  \KOMAoptions{parskip=half}}
\makeatother
\usepackage{xcolor}
\usepackage[margin=1in]{geometry}
\usepackage{graphicx}
\makeatletter
\def\maxwidth{\ifdim\Gin@nat@width>\linewidth\linewidth\else\Gin@nat@width\fi}
\def\maxheight{\ifdim\Gin@nat@height>\textheight\textheight\else\Gin@nat@height\fi}
\makeatother
% Scale images if necessary, so that they will not overflow the page
% margins by default, and it is still possible to overwrite the defaults
% using explicit options in \includegraphics[width, height, ...]{}
\setkeys{Gin}{width=\maxwidth,height=\maxheight,keepaspectratio}
% Set default figure placement to htbp
\makeatletter
\def\fps@figure{htbp}
\makeatother
\setlength{\emergencystretch}{3em} % prevent overfull lines
\providecommand{\tightlist}{%
  \setlength{\itemsep}{0pt}\setlength{\parskip}{0pt}}
\setcounter{secnumdepth}{-\maxdimen} % remove section numbering
\usepackage{graphicx}
\ifLuaTeX
  \usepackage{selnolig}  % disable illegal ligatures
\fi
\IfFileExists{bookmark.sty}{\usepackage{bookmark}}{\usepackage{hyperref}}
\IfFileExists{xurl.sty}{\usepackage{xurl}}{} % add URL line breaks if available
\urlstyle{same} % disable monospaced font for URLs
\hypersetup{
  pdftitle={DWRAT DataScraping GitHub Repository README},
  pdfauthor={Payman Alemi},
  hidelinks,
  pdfcreator={LaTeX via pandoc}}

\title{DWRAT DataScraping GitHub Repository README}
\author{Payman Alemi}
\date{2023-08-01}

\begin{document}
\maketitle

\#README BEGINS HERE

I have set up 3 folders for this project, each of which has several
subfolders. We have intentionally added some of the subfolder paths to
the .gitignore file for this repository because they contain massive
files that GitHub cannot handle. While the files themselves are ignored,
the folders are referenced by Git to maintain the repository's
structural integrity.

\textbf{Supply}\\
This folder contains the files necessary for automating the Santa Rosa
Plains (SRP) GS Flow and PRMS (Precipitation-Runoff Modeling System)
hydrology models. As of 2023-08-01 this folder has 5 subfolders. The
pre-processing and post-processing of the \emph{PRMS} model has been
nearly entirely automated. By contrast, much of the \emph{SRP GS Flow}
model still needs to be automated.

\begin{itemize}
\item
  \textbf{Documentation:} This folder contains relevant emails, PDFs,
  Word Documents and other files containing instructions or information
  about the project.
\item
  \textbf{InputData:} This folder contains the datasets used for loops
  and functions, e.g.~station lists. Maintaining these datasets as CSVs
  is easier than creating then as dataframes. \emph{Gag files in this
  folder are ignored by the .gitignore file.}
\item
  \textbf{ProcessedData:} This folder contains datasets that have been
  manipulated in some way by R Scripts, spreadsheets, etc. \emph{Files
  in this folder are ignored by the .gitignore file.}
\item
  \textbf{Scripts:} This folder contains all scripts associated with
  this project
\item
  \textbf{WebData:} This folder contains unaltered datasets scraped or
  downloaded from the Internet, e.g.~weather station data. \emph{Files
  in this folder are ignored by the .gitignore file.}
\end{itemize}

\textbf{Demand}\\
This folder contains the files necessary for converting the raw diverter
demand data from eWRIMS into a processed, QAQC'd demand dataset ready
for importation into DWRAT. As of 2023-08-01, this folder has 4
subfolders, which just contain placeholder scripts--the actual scripts
have yet to be written.

\begin{itemize}
\item
  \emph{Documentation}: This folder will contain relevant emails, PDFs,
  Word Documents and other files containing instructions or information
  about the project.
\item
  \emph{InputData:} This folder just contains the output CSV from the
  GIS pre-processing steps, which serves as the input file for the R
  pre-processing steps and ultimately the demand module scripts.
\item
  \emph{Intermediate Data:} This folder contains datasets that have been
  modified from the RawData folder. These datasets are considered
  intermediate because they will be used as inputs for other R Scripts.
  \emph{Files in this folder are ignored by the .gitignore file.}
\item
  \emph{Module and Script Comparison}: This folder compares the outputs
  of the Excel Demand Data Modules and the outputs of the corresponding
  R scripts. Each Excel module has a corresponding R Script.
\item
  \emph{OutputData:} This folder will contain the final demand dataset
  CSVs that are ready for importation to the Upper Russian River (URR)
  and Lower Russian River (LRR) Drought Water Rights Allocation (DWRAT)
  models.
\item
  \emph{RawData:} This folder contains the downloaded flat files from
  eWRIMS.\emph{Files in this folder are ignored by the .gitignore file.}
\item
  \emph{Scripts:} This folder will contain the scripts that convert the
  raw diverter demand datasets into the final datasets to be used by the
  DWRAT models.
\end{itemize}

\textbf{Allocation}\\
This folder will contain the files necessary for running the URR and LRR
DWRATs, each of which will have a separate folder.

\begin{itemize}
\tightlist
\item
  \textbf{\emph{URR DWRAT}}

  \begin{itemize}
  \tightlist
  \item
    \emph{Input\textbf{:}} This folder contains the input files for the
    URR DWRAT run.
  \item
    \emph{Output:} This folder contains the output files for the URR
    DWRAT run.
  \end{itemize}
\item
  \textbf{\emph{LRR DWRAT}}

  \begin{itemize}
  \tightlist
  \item
    \emph{Input:} This folder contains the input files for the LRR DWRAT
    run.
  \item
    \emph{Output:} This folder contains the output files for the URR
    DWRAT run.
  \end{itemize}
\end{itemize}

\end{document}
